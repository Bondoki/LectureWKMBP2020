%% LyX 2.3.4.4 created this file.  For more info, see http://www.lyx.org/.
%% Do not edit unless you really know what you are doing.
\documentclass[12pt,ngerman,journal=mamobx,manuscript=article,maxauthors=15,biblabel=plain]{article}
\usepackage[T1]{fontenc}
\usepackage[latin1]{inputenc}
\usepackage[a4paper]{geometry}
\geometry{verbose,tmargin=1in,bmargin=1in,lmargin=1in,rmargin=1in,headheight=0.5cm,headsep=0.5cm,footskip=0.5cm}
\pagestyle{empty}
\setlength{\parskip}{\smallskipamount}
\setlength{\parindent}{0pt}
\usepackage{amsmath}

\makeatletter
%%%%%%%%%%%%%%%%%%%%%%%%%%%%%% User specified LaTeX commands.
\usepackage[ngerman]{babel}
\usepackage{babel}

\makeatother

\usepackage{babel}
\begin{document}

\section*{�bung 1}

\subsection*{Zufallszahlen}

F�r die folgende Aufgabe verwenden Sie den Pseudo-Zufallszahlengenerator
aus der Vorlesung (linearer Kongruenzgenerator). 
\begin{align*}
z_{i+1}=(az_{i}+b)\mod(m)
\end{align*}
Schreiben Sie eine Klasse, mit der Sie folgende Aufgaben erledigen
k�nnen: 
\begin{enumerate}
\item Bestimmen Sie die Periodenl�nge des Pseudo-Zufallszahlengenerators
f�r Werte von $a$, $b$ und $m\in[0,100]$ . 
\item Variieren Sie $a$, $b$ und $m$ in geeigneter Weise und bestimmen
Sie erneut die Periodenl�nge. Was beobachten Sie? Legen Sie optimale
Werte f�r die Parameter als Standardwerte fest (Satz von Knuth). 
\item Betrachten Sie die niedrigsten Bits der Pseudo-Zufallszahlen. Was
beobachten Sie? 
\item Normieren Sie die Zufallszahlen auf das Intervall $[0,1]$. �berlegen
Sie, wie Sie die Korrelation zweier aufeinander folgender normierter
Zufallszahlen mathematisch oder graphisch analysieren k�nnten und
diskutieren Sie ihr Ergebnis/Analyse. 
\end{enumerate}

\subsection*{Monte-Carlo Integration}

Nutzen Sie jetzt einen vorimplementierten Zufallszahlengenerator,
zum Beispiel numpy.random.random.
\begin{enumerate}
\item Verwenden Sie Zufallszahlen um die Kreiszahl $\pi$ zu berechnen.
Betrachten Sie ihre Analyse als Zufallsexperiment und sch�tzen Sie
den Fehler ihrer Auswertung ab. Mit welcher maximalen Genauigkeit
k�nnen Sie die Kreiszahl $\pi$ berechnen? 
\item Betrachten Sie die Funktion $f(x,y)=x^{2}-y^{3}+xy^{2}$. Berechnen
Sie das Volumen, das von dieser Funktion und der $xy$-Ebene f�r Koordinaten
aus den Intervallen $x\in[-1,1]$, $y\in[-1,1]$ eingeschlossen wird.
Stellen Sie das Ergebnis graphisch dar.
\item Beurteilen Sie die Effizienz dieses Integrations-Verfahrens im Vergleich
zu einer gleichm��igen Rasterung des Raumes. Was sind die Vor- bzw.
Nachteile der Monte-Carlo Integration (hochdimensionaler Phasenraum,
periodische Funktionen, ...)? 
\end{enumerate}
N�tzliche Funktionen:
\begin{itemize}
\item numpy.meshgrid, matplotlib.pyplot.plot\_surface
\item numpy.random.random\_sample
\end{itemize}

\end{document}
