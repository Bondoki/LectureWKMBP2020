%% LyX 2.3.4.4 created this file.  For more info, see http://www.lyx.org/.
%% Do not edit unless you really know what you are doing.
\documentclass[12pt,german,journal=mamobx,manuscript=article,maxauthors=15,biblabel=plain,pagestyle=plain]{article}
\usepackage[T1]{fontenc}
\usepackage[latin9]{inputenc}
\usepackage[a4paper]{geometry}
\geometry{verbose,tmargin=1in,bmargin=1in,lmargin=1in,rmargin=1in,headheight=0.5cm,headsep=0.5cm,footskip=0.5cm}
\setlength{\parskip}{\smallskipamount}
\setlength{\parindent}{0pt}
\usepackage{url}
\usepackage{amsmath}
\usepackage{amsthm}
\usepackage{amssymb}

\makeatletter
%%%%%%%%%%%%%%%%%%%%%%%%%%%%%% Textclass specific LaTeX commands.
\numberwithin{equation}{section}
\numberwithin{figure}{section}

\makeatother

\usepackage{babel}
\begin{document}

\section*{�bung 0}

Diese �bungsblatt dient der Einf�hrung in die von mir verwendete Software
jupyter-notebook mit Python 3 als Skriptsprache. Jeder kann selbst
entscheiden, welche Sprache und welches Tool er verwendet, aber f�r
die �bung via Online-Meeting werde ich Python via jupyter-notebook
zeigen. 

\subsection*{Informationen}
\begin{itemize}
\item �bungsleiter Monte-Carlo Simulationen: Martin Wengenmayr (wengenmayr@ipfdd.de)
\item �bungsleiter Molekulardynamik Simulationen: Markus Koch (koch-markus@ipfdd.de)
\item Zugang Materialien (Name: wkmbp, PW: ss2020): \url{https://www.ipfdd.de/de/scmbp/soft-condensed-matter-and-biological-physics/lecture-materials/}
\item Zoom Raum f�r die �bung am 09.04.2020: \url{https://us04web.zoom.us/j/983428740}
\item Alternative: Jitsi-Raum f�r die �bung am 09.04.2020: \url{https://jitsi.tu-dresden.de/UebungNumerikWKMBP}
\item Zur Nutzung von Jitsi gibt es eine ausf�hrliche Seite der TU Dresden:
\url{https://tu-dresden.de/tu-dresden/organisation/zentrale-universitaetsverwaltung/dezernat-3-zentrale-angelegenheiten/sg-3-5-informationssicherheit/tud-cert/videokonferenzservice}
\end{itemize}

\subsection*{Python und jupyter-notebook}

F�r diese �bung verwenden wir die Skriptsprache python. Zum einfacheren
Gebrauch und f�r die unkomplizierte Darstellung �ber die Bildschirmfreigabe
wird jupyter-notebook verwendet. Sollten Sie auf Ihrem Arbeitsger�t
noch keine Python-Distribution und jupyter installiert haben, k�nnen
Sie mit Anaconda schnell und einfach Ihre Arbeitsumgebung aufsetzen: 
\begin{enumerate}
\item Auf \url{www.anaconda.com/distribution/} finden Sie Anaconda zum
herunterladen f�r Windows, macOS und Linux
\item Installieren Sie die Python 3.7 Version nach den Anweisungen des Installers.
\item Testen Sie ihre Installation, indem Sie in dem Ordner, in dem Sie
die �bungen speichern wollen, in einer Shell, der Powershell, der
ZSH, ... den Befehl jupyter-notebook(.exe unter Windows) eingeben. 
\item In Ihrem Standardbrowser sollte sich die jupyter Home Page �ffnen.
Klicken Sie auf der rechten Seite auf New/Neu und w�hlen Sie Notebook:
Python 3 aus.
\end{enumerate}
Lassen Sie sich ``Hallo Welt'' ausgeben. Eine detailierte Anleitung
bis zu diesem Schritt finden sie hier: \url{https://realpython.com/jupyter-notebook-introduction/}

Python ist eine objektorientierte Sprache. Darum werden zu erledigende
Aufgaben in Klassen implementiert, die eigene Variablen (Member) und
eigene (private) Funktionen enthalten. Dieses Prinzip wird in der
folgenden �bung an einem einfachen Beispiel angewendet und wichtige
Bibliotheken werden vorgestellt und benutzt.

\subsection*{Der Calculator}

Schreiben Sie eine Python Klasse mit dem Name ``calculator'', mit
der Sie zwei Zahlen $a$ und $b$ addieren, substrahieren und multiplizieren
k�nnen. Au�erdem soll die Funktion $f(x)=a*x+b$ f�r $x\in\mathbb{N},$
$x<20$ in eine Datei ausgeben werden, sowie eine solche Datei eingelesen
und die Werte von a und b �ber einen Fit bestimmt werden k�nnen. Benutzen
Sie die Bibliotheken numpy, scipy, matplotlib, pandas und os, und
machen sich mit den unten aufgef�hrten Objekten und Funktionen vertraut.

N�tzliche Funktionen und Objekte:
\begin{itemize}
\item Am Anfang des notebooks (Interaktive Umgebung f�r numpy und matplotlib):\textbf{
\%pylab notebook}
\item numpy.array, pandas.DataFrame
\item matplotlib.pyplot.figure und matplotlib.pyplot.plot
\item numpy.loadtxt oder pandas.read\_csv /read\_table
\item numpy.savetxt oder pandas.DataFrame.to\_csv
\item scipy.optimize.curve\_fit
\end{itemize}

\end{document}
